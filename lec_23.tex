% !TEX root = ./Vorlesungsmitschrift AGLA 2.tex  
\lecture{Do 09.07. 10:15}{}
\begin{definition*}
  Sei \( K \) ein Körper, \( V,W,T \) \( K \)-Vektorräume und \( \eta\maps V\times W\to T \) eine bilineare Abbildung. Wir sagen, dass \( \eta \) die universelle Eigenschaft \tensorproperty hat, falls gilt: Für jeden \( K \)-Vektorraum \( U \) und jede bilineare Abbildung \( \varphi\maps V\times W\to U \) gibt es genau eine lineare Abbildung \( f\maps T\to U \), sodass \( \varphi=f\circ \eta \), \dh das Diagramm
  \begin{equation*}
    \begin{tikzcd}
      V\times W\arrow[color=OrangeRed, "\eta", r] \arrow[dr,"\varphi"]&\textcolor{OrangeRed}{T}\arrow[color=LimeGreen,d,"\existsone f"]\\
      &U
    \end{tikzcd}
  \end{equation*}
  kommutiert.
\end{definition*}
\begin{bemerkung*}
  Hat \( \eta\maps V\times W\to T \) die universelle Eigenschaft \tensorproperty, so gilt \( \fieldspan{K}{\image-{\varphi}}=T \).
\end{bemerkung*}
\begin{beispiel*}
  Sei \( V=K \) und \( W \) ein \( K \)-Vektorraum. Dann hat die Abbildung
  \begin{align*}
    \eta\maps K\times W&\to W\\
    (\lambda,w)&\mapsto \lambda\cdot w
  \end{align*}
  die universelle Eigenschaft \tensorproperty. Für eine beliebige bilineare Abbildung \( \varphi\maps K\times W\to U \) setze
  \begin{align*}
    f\maps W&\to U\\
    w&\mapsto \varphi(1,w).
  \end{align*}
  Dann gilt für alle \( \lambda\in K \), \( w\in W \)
  \begin{equation*}
    f(\eta(\lambda,w))=f(\lambda w)=\lambda f(w)=\lambda \varphi(1,w)\explain{\varphi \text{ bilinear}}{=}\varphi(\lambda,w),
  \end{equation*}
  also kommutiert das Diagramm
  \begin{equation*}
    \begin{tikzcd}
      K\times W\arrow[r,"\eta"]\arrow[dr,"\varphi"]&W\arrow[d,"f"]\\
      &U
    \end{tikzcd}
  \end{equation*}
\end{beispiel*}
\begin{lemma}
  Seien \( V,W,T,\tilde{T} \) \( K \)-Vektorräume und \( \eta\maps V\times W\to T \), \( \tilde{\eta}\maps V\times \to \tilde{T} \) bilineare Abbildungen mit der universellen Eigenschaft \tensorproperty.

  Dann gibt es einen Isomorphismus \( f\maps T\to \tilde{T} \) von \( K \)-Vektorräumen mit \( f\circ \eta=\tilde{\eta} \).
\end{lemma}
\begin{proof}
  Nach der universellen Eigenschaft \tensorproperty von \( \eta \) und \( \tilde{\eta} \) gibt es lineare Abbildung
  \begin{equation*}
    f\maps T\to \tilde{T},\quad g\maps \tilde{T}\to T,
  \end{equation*}
  sodass
  \begin{equation*}
    \begin{tikzcd}
      V times W\arrow[r,"\eta"]\arrow[dr,"\tilde{\eta}"]&T\arrow[d,"f"]\\
      &U \arrow[u,bend left,"g"]
    \end{tikzcd}
  \end{equation*}
  kommutiert, \dh \( f\circ \eta=\tilde{\eta} \) und \( \eta=g\circ \tilde{\eta} \), also folgt \( (f\circ g)\circ \tilde{\eta}=\tilde{\eta} \), und das Diagramm
  \begin{equation*}
    \begin{tikzcd}
      V\times W \arrow[r,"\tilde{\eta}"]\arrow[dr,"\tilde{\eta}"]&\tilde{T}\arrow[d,"f\circ g"]\\
      &\tilde{T}
    \end{tikzcd}
  \end{equation*}
  kommutiert. Nach der universellen Eigenschaft \tensorproperty on \( \tilde{\eta} \) folgt \( f\circ g=\Id_{\tilde{T}} \). Ebenso gilt \( f\circ f=\Id_{T} \).
\end{proof}
\begin{frage*}
  Seien \( V,W \) \( K \)-Vektorräume. Gibt es immer eine bilineare Abbildung \( \eta\maps V\times W\to T \) in einem \( K \)-Vektorraum \( T \) mit der universellen Eigenschaft \tensorproperty?
\end{frage*}
\begin{idee*}
  Sind \( V,W \) endlich-dimensionaler Vektorräume mit Basen \( v_1,\dotsc,v_n\in V \) und \( _1,\dotsc,w_m\in W \), dann ist eine bilineare Abbildung \( \varphi\maps V\times W\to U \) eindeutig bestimmt durch die Bilder \( \varphi(v_i,w_j) \) (für einen \( K \)-Vektorraum \( U \)).

  Verwende die Tupel \( \p*{v_{ii}w_j}_{\substack{1\leq i\leq n\\ 1\leq j\leq m}} \) als Basis für den \( K \)-Vektorraum \( T \) mit der bilinearen Abbildung
\begin{align*}
  \eta\maps V\times W&\to T\\
  (v_i,w_j)&\mapsto (v_i,w_j).
\end{align*}
\end{idee*}
\begin{satz}
  Seien \( V,W \) \( K \)-Vektorräume. Dann gibt es einen \( K \)-Vektorraum \( T \) und eine bilineare Abbildung \( V\times W\to T \), welche die universelle Eigenschaft \tensorproperty hat. Falls \( \dim{V}, \dim{W}<\infty \), dann gilt \( \dim{T}=(\dim{V})(\dim{W}) \).
\end{satz}
\begin{proof}
  Sei \( \p*{v_i}_{i\in I} \) Basis von \( V \) und \( \p*{w_j}_{j\in J} \) Basis on \( W \).

  Setze
  \begin{equation*}
    T\definedas \Set{\tau\maps I\times J\to K|\tau(i,j)\neq 0 \text{ für ur endlich viele }(i,j)\in I\times J}.
  \end{equation*}
  Dann ist \( T \) ein \( K \)-Vektorraum unter punktweiser Addition / Multiplikation mit Skalaren
  \begin{gather*}
    (\lambda \tau)(i,j)=\lambda \tau(i,j)\\
    (\tau+\tau')(i,j)=\tau(i,j)+\tau'(i,j)\quad \forall \tau,\tau'\in T\logicspace \forall \lambda\in K\logicspace (i,j)\in I\times J.
  \end{gather*}
  Definiere \( \tau_{i_0,j_0}\in T \) für \( (i_0,j_0)\in I\times J \) durch
  \begin{equation*}
    \tau_{i_0,j_0}(i,j)=\begin{cases}
      1&(i,j)=(i_0,j_0)\\
      0&(i,j)\neq (i_0,j_0).
    \end{cases}
  \end{equation*}
  Dann ist
  \begin{equation*}
    T=\fieldspan{K}{\p*{\tau_{ij}}_{(i,j)\in T\times J}},
  \end{equation*}
  denn für \( \tau\in T \) ist.
  \begin{equation*}
    \tau=\sum_{\mathclap{(i,j)\in I\times J}}\braceannotate{\in K}{\tau(i,j)}\braceannotate{T}{\tau_{ij}}.
  \end{equation*}
  Die Familie \( \p*{\tau_{ij}}_{(i,j)\in I\times J} \) ist linear unabhängig, denn aus
  \begin{equation*}
    \sum_{\mathcal{(i,j)\in I\times J}}'\lambda_{ij}\tau_{ij}=0
  \end{equation*}
  folgt
  \begin{equation*}
    \braceannotate{\lambda_{i',j'}}{\sum_{\mathclap{(i,j)\in I\times J}}'\lambda_{ij}\tau_{ij}(i',j')}=0\quad \forall (i',j')\in I\times J,
  \end{equation*}
  also \( \lambda_{ij}=0\quad \forall (i,j)\in I\times J \). Damit ist \( \p*{\tau_{ij}}_{(i,j)\in I\times J} \) Basis von \( T \). Falls \( \dim{V},\dim{W}<\infty \) gilt insbesondere
  \begin{equation*}
    \dim{T}=(\dim{V})(\dim{W}).
  \end{equation*}
  Sei \( \eta\maps V\times W\to T \) die eindeutig bestimmte lineare Abbildung mit
  \begin{equation*}
    \eta(v_i,w_j)=\tau_{ij}\quad \forall (i,j)\in I\times J.
  \end{equation*}
  \emph{wir zeigen:}\\
  \( \eta\maps V\times W\to T \) hat die universelle Eigenschaft. Sei \( \varphi\maps V\times W\to U \) eine bilineare Abbildung in einen \( K \)-Vektorraum \( U \), setze
  \begin{equation*}
    u_{ij}\definedas \varphi(v_i,w_j)\quad \forall (i,j)\in I\times J
  \end{equation*}
  und sei \( f\maps T\to U \) die eindeutig bestimmte lineare Abbildung mit \( f(\tau_{ij})=u_{ij} \quad \forall (i,j)\in I\times J\). Seien \( v\in  \), \( w\in W \) und schreibe 
  \begin{align*}
    v&={\sum_{i\in I}}'\lambda_i v_i\\
    w&={\sum_{j\in J}}'\mu_j w_j
  \end{align*}
  mit \( \lambda_i,\mu_j\in K \), \( i\in I \), \( j\in J \). Dann gil
  \begin{align*}
    (f\circ \eta)(v,w)&\explain[Big]{\eta \text{ ist bilinear}}{=}f\p*{\sum_{\substack{i\in I\\ j\in J}}\lambda_i \mu_j \eta(v_i,w_j)}\\
    &=f\p*{\sum_{\substack{i\in I\\ j\in J}}\lambda_i \mu_j \tau_{ij}}\\
    &\explain[Big]{f \text{ linear},f(\tau_{ij})=u_{ij}}{=}\sum_{(i,j)\in I\times J}\lambda_i \mu_j u_{ij}\\
    &=\sum_{\mathclap{(i,j)\in I\times J}}\lambda_i \mu_j \varphi(v_i,w_j)\\
    &\explain{\varphi \text{ bilinear}}{=}\varphi(v,w),
  \end{align*}
  also \( f\circ \nu=\varphi \), \dh
  \begin{equation*}
    \begin{tikzcd}
      V\times W\arrow[r,"\eta"]\arrow[dr,"\varphi"]&T\arrow[d,"f"]\\
      &U
    \end{tikzcd}
  \end{equation*}
  kommutiert.

  Ist \( g\maps T\to U \) eine weitere \( K \)-lineare Abbildung \( \varphi=g\circ \eta \), dann gilt
  \begin{equation*}
    u_{ij}=\varphi(v_i,w_j)=g(\tau_{ij})\quad \forall (i,j)\in I\times J,
  \end{equation*}
  also \( f=g \).
\end{proof}
\file{Tensorprodukte Teil 3}
\begin{definition*}
  Seien \( V,W \) \( K \)-Vektorräume. Ein \emph{Tensorprodukt} von \( V \) auf \( W \) ist eine bilineare Abbildung \( \eta\maps V\times W \) in einem \( K \)-Vektorraum \( T \), welche die universelle Eigenschaft \tensorproperty hat.
  
  Schreibe auch
  \begin{equation*}
    \tensorproduct \maps V\times W\to V \fieldtensorproduct{K} W
  \end{equation*}
  für das bis auf Isomorphie eindeutig bestimmte Tensorprodukt von \( V \) und \( W \).

  Für \( v\in V \), \( w\in W \) schreibe \( v\tensorproduct w\) für \( \tensorproduct(v,w) \).
\end{definition*}
\begin{beispiele*}
  \begin{itemize}
    \item Sei \( V=W=\polynomials{K}{t} \). Dann ist
    \begin{align*}
      V\fieldtensorproduct{K} W&=\polynomials{K}{t}\fieldtensorproduct{K} \polynomials{K}{t}\\
      &\simeq \polynomials{K}{t_1,t_2}\\
      \tensorproduct\maps V\times W\to \polynomials{t_1}{t_2}\span.
    \end{align*}
    ist gegeben durch
    \begin{equation*}
      t^i\tensorproduct t^j=t_1^i t_2^j,\quad i,j\geq 0.
    \end{equation*}
    \item Sei \( W=\reals^n \) und \( V=\complexs \), \( K=\reals \). Dann ist die bilineare Abbildung
    \begin{align*}
      \eta\maps \complexs\times \reals^n&\to \complexs^n\\
      (\lambda,\underline{x})&\mapsto \lambda\underline{x} 
    \end{align*}
    ein Tensorprodukt von \( \complexs \) und \( \reals^n \), also
    \begin{equation*}
      \complexs \fieldtensorproduct{\reals} \reals^n=\complexs^n.
    \end{equation*}
    Wir betrachten \( \complexs^n \) als \( 2n \)-dimensionalen Vektorraum mit Basis
    \begin{align*}
      \equalto{1\tensorproduct e_1}{e_1},\dotsc,\equalto{1\tensorproduct e_n}{e_n},\equalto{i\tensorproduct e_1}{ie_1},\dotsc,\equalto{i\tensorproduct e_n}{ie_n}.
    \end{align*}
  \end{itemize}
\end{beispiele*}
\begin{bemerkung*}
  Seien \( V,W \) \( K \)-Vektorräume und \( \tensorproduct \maps V\times W\to V\fieldtensorproduct{K} W \) ein Tensorprodukt. Im Allgemeinen hat nicht jedes Element aus \( V\fieldtensorproduct{K}W \) die Form \( v\tensorproduct w \) für \( \in V \), \( w\in W \). Sei \zb \( V=W \) ein \( K \)-Vektorraum \( 2\leq \dim{V}<\infty \) und \( v_1,v_2,\dotsc,v_n \) Basis von \( V \). Dann ist
  \begin{equation*}
    v_1\tensorproduct v_2+v_2\tensorproduct _1\in V\fieldtensorproduct{K}V.
  \end{equation*}
  Angenommen
  \begin{equation*}
    v_1\tensorproduct v_2+v_2\tensorproduct v_1=v\tensorproduct w
  \end{equation*}
  mit \( v,w\in V \). Schreibe \( v=\sum_{i=1}^{n}\lambda_i v_i \), \( w=\sum_{j=1}^{n}\mu_j w_j \). Dann ist
  \begin{align*}
    v_1\tensorproduct _2+v_2\tensorproduct v_1=\tensorproduct \p*{\sum_{i=1}^{n}\lambda_i v_i, \sum_{j=1}^{n}\mu_j v_j}\\
    &=\sum_{i=1}^{n}\sum_{j=1}^{n}\lambda_i \mu_j v_i\tensorproduct v_j.
  \end{align*}
Die Elemente \( v_i\tensorproduct v_j \), \( 1\leq i,j\leq n \), bilden eine Basis von \( V\tensorproduct W \), also folgt nach Koeffizientenvergleich \( \lambda_1 \mu_1=0 \), \( \lambda_1\mu_2=1 \), \( \lambda_2\mu_1=1 \), \( \lambda_2 \mu_2=0 \), \contra.
\end{bemerkung*}
\subsection*{Weitere Eigenschaften des Tensorprodukts}
\begin{lemma}
  Seien \( V,W \) \( K \)-Vektorräume mit Tensorprodukt \( \tensorproduct\maps V\times W\to V\fieldtensorproduct{K}W \). Dann gilt für \( v,v'\in V \), \( w,w'\in W \), \( \lambda\in K \), dass 
  \begin{enumerate}
    \item \label{tensorprodukt:distributiv}
    \begin{align*}
      v\tensorproduct w+v'\tensorproduct w=(v+v')\tensorproduct w\\
      v\tensorproduct w+\tensorproduct w'=v\tensorproduct (w+w')
    \end{align*}
    \item \( (\lambda\cdot v)\tensorproduct w=v\tensorproduct (\lambda w)=\lambda (v \tensorproduct w) \).
  \end{enumerate}
\end{lemma}
\begin{proof}
  Folgt aus der Bilinearität der Abbildung \( \tensorproduct(v,w)=v\tensorproduct w \).
\end{proof}
\begin{lemma}
  Seien \( V,W \) \( K \)-Vektorräume mit Tensorprodukt
  \begin{equation*}
    \tensorproduct\maps V\times W\to V\fieldtensorproduct{K} W,
  \end{equation*}
  \( v_1,\dotsc,v_n\in V \), \( w_1,\dotsc,w_n\in W \) und \( v_1,\dotsc,v_n \) linear unabhängig mit
  \begin{equation*}
    \sum_{i=1}^{n}v_i\tensorproduct w_i=0
  \end{equation*}
  in \( V\fieldtensorproduct{K}W \). Dann gilt \( w_0=0 \), \( 1\leq i\leq n \).
\end{lemma}
\begin{proof}
  Für \( 1\leq i\leq n \) definiere lineare Abbildungen \( f_i\maps V\to K \) mit \( f_i(v_j)=\kroneckerdelta{ij} \), \( 1\leq i,j\leq n \). Seien \( g_i\maps W\to K \), \( 1\leq i\leq n \), beliebige lineare Abbildungen. Setze 
  \begin{equation*}
    \varphi(v,w)\definedas \sum_{i=1}^{n}f_i(v)g_i(w).
  \end{equation*}
  Dann ist \( \varphi\maps V\times W\to K \) bilinear, also \texistsone lineare Abbildung \( h\maps \fieldtensorproduct{K}W\to K \) mit \( \phi=h\circ \tensorproduct \).
  \begin{equation*}
    \begin{tikzcd}
      V\times W\arrow[r,"\tensorproduct"]\arrow[dr,"\varphi"]&V\fieldtensorproduct{K}W\arrow[d,"h"]\\
      &K,
    \end{tikzcd}
  \end{equation*}
  \dh
  \begin{align*}
    0&=h\p[Big]{\braceannotate{=0}{\sum_{i=1}^{n}v_i\tensorproduct w_i}}\\
    &=\sum_{i=1}^{n}h(v_i\tensorproduct w_i)\\
    &=\sum_{i=1}^{n}\phi(v_i,w_i)\\
    &=\sum_{i=1}^{n}\sum_{j=1}^{n}\braceannotate{\kroneckerdelta{ij}}{f_j(_i)}g_j(w_i)\\
    &=\sum_{i=1}^{n}g_i(w_i)\tag{\( * \)}\label{eq:funktion_gleich_null_wenn_tensor_gleich_null}.
  \end{align*}
  Da \eqref{eq:funktion_gleich_null_wenn_tensor_gleich_null} für alle linearen Abbildungen
  \begin{equation*}
    g_i\maps W\to K,\quad 1\leq i\leq n
  \end{equation*}
  gilt, folgt \( w_i=0 \), \( 1\leq i\leq n \).
\end{proof}
\begin{korollar}
  Seien \( V,W \) \( K \)-Vektorräume mit Tensorprodukt
  \begin{equation*}
    \tensorproduct\maps V\times W\to V\fieldtensorproduct{K}W 
  \end{equation*}
  und \( v\in V\setminus \zeroset \), \( w\in W\setminus \zeroset  \). Dann ist \( v\tensorproduct w\neq 0 \) in \( V\tensorproduct W \).
\end{korollar}
\subsection*{Eine alternative Konstruktion des Tensorprodukts}
Seien \( V,W \) \( K \)-Vektorräume. Definiere
\begin{equation*}
  \freevectorspace{V\times W}\definedas \bigoplus_{(v,w)\in V\times W}K\cdot \gamma_{(v,w)},
\end{equation*}
den freien Vektorraum erzeugt durch \( (v,w)\in V\times W \). Sei \( N(V,W)\subseteq F(V\times W) \) der Untervektorraum aufgespannt durch die Elemente
\begin{gather*}
  \gamma_{(\lambda v_1+\mu v_2,w)}-\lambda \gamma_{(v_1,w)}-\mu \gamma_{(v_2,w)}\\
  \gamma_{(v,\lambda w_1+\mu w_2)}-\lambda \gamma_{(v,w_1)}-\mu \gamma_{v,w_2}
\end{gather*}
für \( \lambda,\mu\in K \), \( v_1,_2\in V \), \( w,w_1+w_2\in W \). Sei
\begin{equation*}
  T\definedas\quotient{\freevectorspace{V\times W}}{N(V,W)}
\end{equation*}
und
\begin{equation*}
  \pi \definedas \freevectorspace{V\times W}\to T
\end{equation*}
die Projektionsabbildung.

Definiere
\begin{align*}s
  \eta\maps V\times W&\to T\\
  (v,w)&\mapsto \pi(\gamma_{(v,w)}).
\end{align*}
\begin{behauptung}
  \( \eta \) ist bilinear.
\end{behauptung}
Sei \( w\in W \) fest, \( v_1,v_2\in V \), \( \lambda,\mu\in K \). Dann gilt
\begin{align*}
  \eta(\lambda v_1+\mu v_2,w)&=\pi\p*{\gamma_{(\lambda v_1+\mu v_2 ,w)}}\\
  &=\pi\p[Big]{\braceannotate{\in N(V,W)}{\gamma_{(\lambda v_1+\mu v_2,w)}-\lambda \gamma_{(v_1,w)}-\mu\gamma_{(v_2,w)}}+\lambda \gamma_{(v_1,w)}+\mu\gamma_{(v_2,w)}}\\
  &=\pi\p*{\lambda \gamma_{(v_1,w)}+\mu\gamma_{(v_2,w)}}
  &=\lambda\pi\p*{\gamma_{(v_1,w)}}+\mu \pi\p*{\gamma_{(v_2,w)}}\\
  &=\lambda \eta(v_1,w)+\mu \eta(v_2,w).
\end{align*}
Ebenso ist \( \eta \) im zweiten Argument linear.
\begin{behauptung*}
  Die Abbildung
  \begin{equation*}
    \eta\maps V\times W\to T
  \end{equation*}
  hat die universelle Eigenschaft \tensorproperty. Sei \( \varphi\maps V\times W\to U \) eine bilineare Abbildung in einen \( K \)-Vektorraum \( U \). Sei
  \begin{equation*}
    g\maps \freevectorspace{V\times W}\to U
  \end{equation*}
  die eindeutig bestimmte lineare Abbildung mit
  \begin{equation*}
    g\p*{\gamma_{(v,w)}}=\phi(v,w)\quad \forall (v,w)\in V\times W.
  \end{equation*}
  Da \( \varphi \) bilinear ist, folgt
  \begin{equation*}
    N(V,W)\subseteq \Kern{g}.
  \end{equation*}
  Also induziert \( g  \) eine Abbildung
  \begin{equation*}
    \bar{g}\maps \quotient{\freevectorspace{V\times W}}{N(V,W)}\to U
  \end{equation*}
  und das Diagramm
  \begin{equation*}
    \begin{tikzcd}
      V\times W\arrow["\eta",r]\arrow[dr,"\varphi"]& \overbrace{\quotient{\freevectorspace{V\times W}}{N(V,W)}}^T\arrow["\bar{g}",d]\\
      &U
    \end{tikzcd}
  \end{equation*}
  kommutiert, denn für \( (v,w)\in V\times W \) ist
  \begin{equation*}
    \bar{g}(\eta(v,w))=\bar{g}\p*{\pi\p*{\gamma_{(v,w)}}}=\gamma\p*{\gamma_{(v,w)}}=\varphi(v,w).
  \end{equation*}
  Sei \( f\maps T\to U \) eine weiter \( K \)-lineare Abbildung mit \( f\circ \eta=\varphi \). Dann ist
  \begin{equation*}
    \phi(v,w)=f\p*{\pi\p*{\gamma_{(v,w)}}}\quad \forall (v,w)\in V\times W.
  \end{equation*}
  Die Elemente \( \pi\p*{\gamma_{(v,w)}} \) erzeugen \( T \) für \( (v,w)\in V\times W \), also ist \( f \) eindeutig bestimmt.
\end{behauptung*}