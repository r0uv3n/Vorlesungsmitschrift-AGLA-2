% !TEX root = ./Vorlesungsmitschrift AGLA 2.tex  
\lecture{Di 05.05. 10:15}{}
\begin{beispiel*}
    Seien \( p_0,p_1\in \reals^2 \), \( p_0\neq p_1 \). Ziel: Beschreibe den affinen Unterraum \( p_0\vee p_1 \) als Teilmenge des \( \reals^2 \). Sei \( p\in p_0\vee p_1 \). Dann \( \exists \lambda\in \reals \) mit \( \vv{p_0 p}=\lambda \vv{p_0 p_1} \) und als Vektoren im \( \reals^2 \) gilt \( p=p_0+\lambda(p_1-p_0) \). Es gilt
    \begin{align*}
        p_0 \vee p_1=\Set{(1-\lambda)p_0+\lambda p_1,\logicspace \lambda\in \reals}.
    \end{align*}
    \begin{figure}[H]
        \centering
        \includegraphics[width=0.5\linewidth]{figures/affine_verbindungsgerade}
        \label{fig:affine_verbindungsgerade}
    \end{figure}
    
\end{beispiel*}
\begin{frage*}
    Verallgemeinerung zu höherdimensionalen Räumen?
\end{frage*}
\begin{definition*}
    Seien \( p_0,\dotsc,p_k\in  K^n \). Wir nennen eine Linearkombination
    \begin{align*}
        \lambda_0 p_0+\lambda_1 p_1+\dotsb+\lambda_m p_m
    \end{align*}
    mit \( \lambda_i\in K \), \( 0\leq i\leq m \) eine Affinkombination oder affin falls gilt \( \lambda_0+\lambda_1+\dotsb+\lambda_m=1 \).
\end{definition*}
\begin{satz}
    Seien \( p_0,\dotsb,p_m\in K^n \). Dann gilt
    \begin{align*}
        p_0\vee \dotsb \vee p_m =\Set{\sum_{i=0}^{m}\lambda_i p_i\in K^n \logicspace \lambda_0,\dotsc,\lambda_m\in K, \sum_{i=0}^{m}\lambda_i=1}.
    \end{align*}
\end{satz}
\begin{proof}
    Sei \( Y=p_0 \vee\dotsb\vee p_m\in K^n \). Es gilt
    \begin{align*}
        T(Y)\begin{aligned}[t]
            &=\underbrace{T(p_m)}_{=0}+T(p_0\vee \dotsb \vee p_{m-1})+\underbrace{K\vv{p_0 p_m}}_{\mathclap{=T(p_0\vee p_m)}}\\
            &=K\vv{p_0 p_m}+T(p_0\vee \dotsb\vee p_{m-1})\\
            &=K\vv{p_0 p_m}+\dotsb +K\vv{p_0 p_1}\\
            &\vdots\\
            &=K\vv{p_0 p_m}+\dotsb + K\vv{p_0 p_1}\\
            &=\span(\vv{p_0 p_1}, \dotsc, \vv{p_0 p_m}).
        \end{aligned}
    \end{align*}
    Sei \( p\in K^n \). Dann ist \( p\in Y \) genau dann, wenn \texists \( \lambda_1,\dotsc,\lambda_m\in K \) mit
    \begin{align*}
        \vv{p_0 p}=\lambda_1 \vv{p_0 p_1}+\dotsb+\lambda_m\vv{p_0 p_m}.
    \end{align*}
    Im \( K^n \) gilt dann also
    \begin{align*}
        p-p_0=\lambda_1(p_1-p_0)+\dotsb + \lambda_m(p_m-p_0)
    \end{align*}
    oder 
    \begin{align*}
        p=\lambda_0 p_0+\lambda_1 p_1+\dotsb+\lambda_m p_m
    \end{align*}
    mit \( \lambda_0=1-\lambda_1-\dotsb-\lambda_m \), \dh \( \sum\limits_{i=0}^{m}\lambda_i=1 \).
\end{proof}
\section{Affine Abbildungen und Matrizen, Fixpunkte}
\begin{motivation*}
    Seien \( V,W \) \( K \)-Vektorräume, \( F\maps V\to W \) eine lineare Abbildung. Wenn wir für \( V \) und \( W \) Basen wählen, dann können wir die Abbildung \( F \) eindeutig durch eine Matrix beschreiben.
\end{motivation*}
\begin{frage*}
    Inwiefern können wir affin Abbildung zwischen affinen Räumen durch Matrizen beschreiben?
\end{frage*}
Wahl von Basen in Vektorräumen \( \leftrightarrow \) Wahl von Koordinaten in affinen Räumen. 

Seien \( X,Y \) affine Räume über \( K \), \( f\maps X\to Y \) eine affine Abbildung. Wähle affine Koordinatensysteme \( \phi\maps K^n\to X  \) und \( \psi\maps K^m\to Y \).

Wir haben das folgende kommutative Diagramm
\begin{equation*}
    \begin{tikzcd}
        K^n\arrow{r}{\phi}\arrow{d}[name=g]{g} &X\arrow{d}[name=f]{f}\\
        K^m\arrow{r}{\psi}&Y
        \arrow[to path={(f) node[midway,scale=1] {\rotatebox{90}{\(\circlearrowright\)}} (g)}]{} 
    \end{tikzcd}    
\end{equation*}
mit \( g=\inv{\psi}\circ f\circ \phi \) affin. \( g \) ist affin, also besteht eine affine Abbildung \( G\maps K^n\to K^m \) mit
\begin{align*}
    g(x)-g(0)=G(x)\quad \forall x\in K^n.
\end{align*}
\( G \) ist linear, also können wir \( G \) durch eine Matrix \( A \) ausdrücken.
\begin{align*}
    g(x)=Ax+b\quad \forall x\in K^n.
\end{align*}
mit \( b=g(0) \).
\begin{frage*}
    Wie können wir \( A \) berechnen gegeben eine affine Basis \( (p_0,\dotsc,p_n) \) von \( K^n \) und \( g(p_i) \), \( 0\leq i \leq n \)?
\end{frage*}
\begin{figure}[H]
    \centering
    \includegraphics[width=0.5\linewidth]{figures/affine_basen_abbildung_wunsch}
    \label{fig:affine_basen_abbildung_wunsch}
\end{figure}
Wir betrachten die Matrizen \( B\in \Mat_{m\times n}(K) \) bestehend aus den Spaltenvektoren \( \vv{q_0 q_1},\dotsc, \vv{q_0 q_n} \) und \( S\in \Mat_{n\times n}(K) \) bestehend aus den Spaltenvektoren \( \vv{p_0 p_1}, \dotsc, \vv{p_0 p_n} \). Dann gilt \( A=B\cdot \inv{S}  \) und \( g(x)-g(p_0)=A(x-p_0) \), also \( g(x)=Ax+b \) mit \( b=g(p_0)-Ap_0 \).
\begin{bemerkung*}
    Wählen wir für \( p_0,\dotsc,pm \) die affine Basis \( 0,e_1,\dotsc, e_n \), dann \( S=\Id_{n\times n} \) und \( A=B \).
\end{bemerkung*}
\subsection*{Fixpunkte}
\begin{beispiel}
    Betrachte die affine Abbildung \( f\maps K\to K \), \( K \) ein Körper, in der Matrizendarstellung gegeben durch \( f(x)=2x+1\isittrue{=}x \).
    \begin{figure}[H]
        \centering
        \includegraphics[width=0.5\linewidth]{figures/affiner_fixpunkt_1_d}
        \label{fig:affiner_fixpunkt_1_d}
    \end{figure}
    Dann gibt es genau ein \( x\in K \) mit \( f(x)=x \), nämlich \( x=-1 \).
\end{beispiel}
\begin{definition*}
    Sei \( X \) ein affiner Raum \( f\maps X\to X \) eine affine Abbildung. Wir nennen
    \begin{align*}
        \Fix(f)\definedas \Set{x\in X|f(x)=x}
    \end{align*}
    die Menge der Fixpunkte von \( f \).
\end{definition*}
\begin{frage*}
    Welche Struktur hat \( \Fix(f) \).
\end{frage*}
\begin{beispiel}
    \( X \) affiner Raum.
    \begin{align*}
        \Id\maps \begin{aligned}[t]
            X&\to X\\
            x&\mapsto x
        \end{aligned}
    \end{align*}
    dann \( \Fix(\Id)=X \).
\end{beispiel}
\begin{beispiel}
    \( f\maps K^n\to K^n \), \( x\mapsto \underrelate{\isittrue{=}}{x}{\underbrace{x+p_0}} \) mit \( p_0\in K^n\setminus \zeroset \), dann \( \Fix(f)=\emptyset \).
\end{beispiel}
\begin{beispiel}
    \begin{frage*}
        Was sind die Fixpunkte einer Projektion?
    \end{frage*}
    
\end{beispiel}
\begin{lemma}\label{affine_fixpunkte_sind_affiner_unterraum}
    \( \Fix(f)\subseteq X \) ist ein affiner Unterraum.
\end{lemma}
\begin{proof}
    Falls \( \Fix(f)=\emptyset \) dann \checkmark. Sei also \( \Fix(f)\neq \emptyset \) und \( p\in \Fix(f) \), \( F \) die zu \( f \) gehörig lineare Abbildung.

    Für \( x\in \Fix(f) \) gilt
    \begin{align*}
        \vv{px}=\vv{f(p) f(x)}=F(\vv{px}).
    \end{align*}
    Umgekehrt folgt aus
    \begin{align*}
        \vv{px}=F(\vv{px})=\vv{p f(x)},
    \end{align*}
    dass \( x=f(x) \), also \( x\in \Fix(f) \).

    Damit gilt
    \begin{align*}
        \Set{\vv{px}\in T(X)|x\in \Fix(f)}=\Set{\vv{px}\in T(X)|\vv{px}=F(\vv{px})}
    \end{align*}
    und wir erkennen diese Menge als \( K \)-Untervektorraum von \( X \).
\end{proof}
\begin{frage*}
    Bestimmung von \( \Fix(f) \) für eine beliebige affine Abbildung \( f\maps X\to X \)?
\end{frage*}
Nach Wahl eines Koordinatensystems können wir auf den Fall \( X=K^n \) reduzieren und annehmen, dass \( f \) in Matrizendarstellung gegeben ist.

Sei also
\begin{align*}
    f\maps \begin{aligned}[t]
        K^n&\to K^n\\
        x\mapsto &\underbrace{Ax+b}_{=x=\Id_n x}.
    \end{aligned}
\end{align*}
Dann gilt
\begin{align*}
    \Fix(f)=\set{x\in K^n|(A-\explain{\text{Einheitsmatrix der Dimension \( n \):}\ \begin{pNiceMatrix}
        1 &  & 0 \\
         & \ddots &  \\
        0 &  & 1
    \end{pNiceMatrix}
    }{\Id_n})x=-b}
\end{align*}
Wir haben das Problem also reduziert auf das Lösen eines linearen Gleichungssystems.
\begin{bemerkung*}
    Daraus kann man auch \thref{affine_fixpunkte_sind_affiner_unterraum} ableiten.
\end{bemerkung*}
\begin{beispiel}\label{dilatation_beispiel}
    \begin{align*}
        f\maps \begin{aligned}[t]
            K^n&\to K^n\\
            x&\mapsto \lambda \Id_n x+b
        \end{aligned}     
    \end{align*}
    mit \( \lambda\in K \).

    Dann
    \begin{align*}
        \Fix(f)=\Set{x\in K^n|(\lambda-1)x=-b}.
    \end{align*}
    Falls \( \lambda-1 \) invertierbar ist \( (\lambda\neq 1) \), gibt es genau einen Fixpunkt.
\end{beispiel}
\begin{definition*}
    Sei \( f\maps X\to X \) eine affine Abbildung mit zugehöriger linearer Abbildung \( F\maps T(X)\to T(X) \). Wir nennen \( f \) eine \emph{Dilatation} mit \emph{Faktor \( \lambda \)}, falls gilt
    \begin{align*}
        F=\lambda \cdot \Id_{T(X)}\quad \lambda\in K.
    \end{align*}
    Im Fall \( \lambda=1 \) nennen wir \( f \) eine Translation.
\end{definition*}
\begin{lemma}
    Sei \( f\maps X\to X \) eine Dilatation mit Faktor \( \lambda\neq 1 \). Dann gilt
    \begin{align*}
        \anzahl \Fix(f)=1.
    \end{align*}
\end{lemma}
\begin{proof}
    Nach Wahl eines Koordinatensystems reduzieren wir das Problem auf \thref{dilatation_beispiel}.    
\end{proof}

\section{Kollineationen}
Sei \( f\maps X\to X \) eine affine Abbildung eines affinen Raumes \( X \), \zb eine Affinität. Seien \( p_1,p_2,p_3\subset X \) in einer Geraden \( \ell\subseteq X \) enthalten.
\begin{figure}[H]
    \centering
    \includegraphics[width=0.5\linewidth]{figures/kollineationen_motivation}
    \label{fig:kollineationen_motivation}
\end{figure}
Dann liegen auch \( f(p_1), f(p_2),f(p_3) \) auf einer Geraden.

\begin{frage*}
    Welche bijektiven Abbildungen \( f\maps X\to X \) haben diese Eigenschaft?
\end{frage*}
\begin{definition*}
    Sei \( X \) ein affiner Raum und \( p_1,p_2,p_3\in X \). Wir nennen \( p_1,p_2,p_3 \) \emph{kollinear}, wenn \( p_1,p_2,p_3 \) auf einer Geraden \( \ell \subset X \) liegen. Wir nennen eine bijektive Abbildung \( f\maps X\to X \) eine Kollineation, falls jede Gerade \( \ell \subset X \) auf eine Gerade \( f(\ell)\subset X \) abgebildet wird.
\end{definition*}
\begin{beispiel}
    Affinitäten
\end{beispiel}
\begin{beispiel}
    Ist \( \dim X=1 \) und \( f\maps X\to X \) bijektiv, dann ist \( f \) eine Kollineation.
\end{beispiel}
\begin{beispiel}\label{kollinieationen:beispiele:komplexe_konjugation}
    Sei \( X=\complexs^2 \) als affiner Raum über \( \complexs \).
    \begin{align*}
        f\maps \begin{aligned}[t]
            \complexs^2\to \complexs^2\\
            (x,y)&\mapsto &(\explain{\text{komplexe Konjugation}}{\conj{x},\conj{y          }}).
        \end{aligned}
    \end{align*}
    Dann ist \( f \) eine Kollineation. Das Bild einer Geraden
    \begin{align*}
        (x_0,y_0)+\complexs(x_1,y_1)
    \end{align*}
    ist gegeben durch die Gerade
    \begin{align*}
        (\conj{x_0},\conj{y_0})+\complexs(\conj{x_1},\conj{y_1}),
    \end{align*}
    aber \( f \) ist \emph{keine Affinität}!
\end{beispiel}
\begin{bemerkung*}
    Die komplexe Konjugation
    \begin{align*}
        \begin{aligned}[t]
            \complexs&\to \complexs\\
            x&\mapsto \conj{x}
        \end{aligned}
    \end{align*}
    ist ein Automorphismus von dem Körper \( \complexs \).
\end{bemerkung*}
\begin{definition*}
    Sei \( K \) ein Körper. Wir nennen eine Bijektion \( \alpha\maps K\to K \) einen Automorphismus von \( K \) falls gilt
    \begin{align*}
        \alpha(\lambda+\mu)&=\alpha(\lambda)+\alpha(\mu)\quad \forall \lambda,\mu\in K
        \intertext{und}
        \alpha(\lambda\cdot \mu)&=\alpha(\lambda)\cdot \alpha(\mu)\quad \forall  \lambda,\mu\in K
    \end{align*}
\end{definition*}
\begin{beispiel}
    \begin{align*}
        K=\rationals(\sqrt{2})=\Set{x+y\sqrt{2}|x,y\in\rationals}
    \end{align*}
    ist ein Körper und 
    \begin{align*}
        \alpha\maps \begin{aligned}[t]
            \rationals(\sqrt{2})&\to \rationals(\sqrt{2})\\
            x+y\sqrt{2}&\mapsto x-y\sqrt{2}.
        \end{aligned}
    \end{align*}
\end{beispiel}
\begin{satz}
    Sei \( \alpha\maps \reals\to \reals \) ein Automorphismus von \( \reals \). Dann gilt \( \alpha=\Id_{\reals} \).
\end{satz}
\begin{proof}
    Sei \( \alpha\maps \reals\to \reals \) ein Automorphismus.
    \begin{proofenumerate}
        \item Dann gilt
        \begin{align*}
            \alpha(0)=\alpha(0+0)=\alpha(0)+\alpha(0),
        \end{align*}
        also \( \alpha(0)=0 \).
        \item Dann gilt
        \begin{align*}
            0=\alpha(0)=\alpha(\lambda-\lambda)=\alpha(\lambda)+\alpha(-\lambda),
        \end{align*}
        also \( \alpha(-\lambda)=-\alpha(\lambda)\logicspace \forall \lambda\in \reals \).
        \item Dann gilt
        \begin{align*}
            \alpha(1)=\alpha(1\cdot 1)=\alpha(1)\alpha(1),
        \end{align*}
        also \( \alpha(1)=1 \) und daher
        \begin{align*}
            \alpha(n)=n\logicspace \forall n\in \wholes,
        \end{align*}
        \zb
        \begin{align*}
            \alpha(2)=\alpha(1+1)=\alpha(1)+\alpha(1)=1+1=2.
        \end{align*}
        \item Sei \( p\in \wholes \), \( q\in \naturals \), dann gilt
        \begin{align*}
            q\alpha\left( \frac{p}{q} \right)\begin{aligned}[t]
                &=\alpha(q)\alpha\left( \frac{p}{q} \right)=\alpha\left( q\frac{p}{q} \right)=\alpha(p)=p,
            \end{aligned}
        \end{align*}
        also \( \alpha\left( \frac{p}{q}=\frac{p}{q} \right) \) oder \( \alpha(t)=t\quad \forall t\in \rationals \).
        \item Sei \( \lambda\in \reals_{>0} \). Dann \texists  \( \mu\in\reals \) mit \( \lambda=\mu^2 \) und
        \begin{align*}
            \alpha(\lambda)=\alpha(\mu^2)=\alpha(\mu)\cdot \alpha(\mu)>0,
        \end{align*}
        also
        \begin{align*}
            \alpha(\lambda)>0\quad \forall \lambda\subset \reals>0.
        \end{align*}
    \end{proofenumerate}
    Wir zeigen nun \( \alpha(\lambda)=\lambda\quad \forall \lambda\in \reals \).

    \minisec{Gegenannahme}
    Sei \( \lambda\in \reals \) mit \( \alpha(\lambda)\neq \lambda \). Wir diskutieren den Fall \( \alpha(\lambda)<\lambda \) (\( \alpha(\lambda)>\lambda \) geht genauso).

    Wähle \( \frac{p}{q}\in \rationals \) mit
    \begin{align*}
        \alpha(\lambda)<\frac{p}{q}<\lambda.
    \end{align*}
    Dann gilt
    \begin{align*}
        \alpha(\lambda-\frac{p}{q})=\alpha(\lambda)-\frac{p}{q}<0
    \end{align*}
    \contra zu \( \lambda-\frac{p}{q}>0 \).
\end{proof}

\subsection*{Eine Familie von Kollineationen}
\begin{idee*}
    Wir verallgemeinern \thref{kollinieationen:beispiele:komplexe_konjugation}, um ine größere Klasse an Kollineationen zu erhalten als Affinitäten.
\end{idee*}
\begin{beispiel}
    \begin{align*}
        f\maps \begin{aligned}[t]
            \complexs^2&\to \complexs^2\\
            (x,y)&\mapsto (\conj{x},\conj{y})
        \end{aligned}
    \end{align*}
    respektiert Addition, \dh
    \begin{align*}
        f(z+z')=f(z)+f(z')\quad \forall z,z'\in \complexs^2,
    \end{align*}
    und hat die Eigenschaft
    \begin{align*}
        f(\lambda z)=\conj{\lambda} f(z)\quad \forall \lambda\in \complexs\logicspace \forall z\in \complexs^2.
    \end{align*}
    \tto Wir nennen \( f \) semilinear.
\end{beispiel}
\begin{definition*}
    Seien \( V,W \) Vektorräume über einem Körper \( K \). Wir nennen eine Abbildung \( F\maps V\to W \) \emph{semilinear}, wenn es einen Automorphismus \( \alpha \) von \( K \) gibt, sodass gilt
    \begin{itemize}
        \item \(F(v+v')=F(v)+F(v')\quad \forall v,v'\in V\)
        \item \( F(\lambda v)=\alpha(\lambda) F(v)\quad \forall \lambda\in K \logicspace \forall v\in V\).
    \end{itemize}
\end{definition*}
\begin{definition*}
    Seien \( X,Y \) affine Räume über einem Körper \( K \). Wir nennen eine Abbildung
    \begin{align*}
        f\maps X\to Y
    \end{align*}
    \emph{semiaffin}, wenn es eine \emph{semilineare Abbildung} \( F\maps T(X)\to T(Y) \) gibt mit
    \begin{align*}
        \vv{f(p) f(q)}=F(\vv{pq})\logicspace \forall p,q\in X.
    \end{align*}
    Falls \( f \) außerdem bijektiv ist, dann nennen wir \( f \) eine Semiaffinität.
\end{definition*}