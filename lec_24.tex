% !TEX root = ./Vorlesungsmitschrift AGLA 2.tex  
\lecture{Di 14.07. 10:15}{}
\subsection*{Eine alternative Konstruktion des Tensorprodukts}
Seien \( V,W \) \( K \)-Vektorräume. Definiere
\begin{equation*}
  F(V\times W)\definedas \bigoplus_{(v,w)\in V\times W}K\cdot \gamma_{(v,w)},
\end{equation*}
den freien Vektorraum erzeugt durch \( (v,w)\in V\times W \). Sei \( N(V,W)\subseteq F(V\times W) \) der Untervektorraum aufgespannt durch die Elemente.  
\begin{gather*}
  \gamma_{(\lambda v_1+\mu v_2,w)}-\lambda \gamma_{(v_1,w)}-\mu \gamma_{(v_2,w)}\\
  \gamma_{(v,\lambda w_1+\mu w_2)}-\lambda \gamma_{(v,w_1)}-\mu \gamma_{v,w_2}
\end{gather*}
für \( \lambda,\mu\in K \), \( v_1,_2\in V \), \( w,w_1+w_2\in W \). Sei