% !TEX root = ./Vorlesungsmitschrift AGLA 2.tex  
\lecture{Fr 29.05. 10:15}{}
Im \thref{projektive_unterebene} haben wir angedeutet, dass sich zwei Geraden im \( \projectionspaceover{2}{\reals} \) immer schneiden. Ganz allgemein gilt folgender Satz.
\begin{satz}[Dimensionsformel]
  Sei \( V \) ein \( K \)-Vektorraum und \( Z_1,Z_2\subseteq \projectionspace{V} \) projektive Unterräume. Dann gilt
  \begin{equation*}
    \projectivedim-{Z_1 \vee Z_2}=\projectivedim-{Z_1}+\projectivedim-{Z_2}-\projectivedim{Z_1\cap Z_2}.
  \end{equation*}
  Falls \( \projectivedim-{Z_1}+\projectivedim-{Z_2}\geq\projectivedim-{\projectionspace{V}} \), dann gilt \( Z_1\cap Z_2\neq \emptyset \).
\end{satz}
\begin{proof}
  Sei \( Z_i=\projectionspace{W_i} \), \( 1\leq i\leq 2 \) mit \( W_1, W_2\untervektorraum V\) \( K \)-Untervektorräu me. Es gilt dann
  \begin{align*}
    \projectivedim{Z_1 \vee Z2}&\explain{\text{\thref{projektiver_verbindungsraum_formel}}}{=}\projectivedim{\projectionspace{W_1+W_2}}\\
    &=\fielddim{K}{W_1+W_2}-1\\
    &\explain{\text{Dimensionsformel für Untervektorräume aus der \aglacourse{1}}}{=}\fielddim-{K}{W_1}+\fielddim-{K}{W_2}-\fielddim-{K}{W_1\cap W_2}\\
    &=(\fielddim{K}{W_1}-1)+(\fielddim{K}{W_2}-1)-(\fielddim{K}{W_1\cap W_2}-1)\\
    &\explain{\text{Beweis von \thref{schnitt_von_projektiven_unterraeumen_ist_projektiver_unterraum}}}{=}\projectivedim-{\projectionspace{W_1}}+\projectivedim-{\projectionspace{W_2}}-\projectivedim-{\underbrace{\projectionspace{W_1}\cap\projectionspace{W_2}}_{=\projectionspace{W_1\cap W_2}}}\\
    &=\projectivedim-{Z_1}+\projectivedim-{Z_2}-\projectivedim-{Z_1\cap Z_2}.
  \end{align*}
  Ist 
  \begin{equation*}
    \projectivedim-{Z_1}+\projectivedim-{Z_2}\geq \projectivedim{\projectionspace{V}}\leq \projectivedim{Z_1\vee Z_2}
  \end{equation*}
  dann gilt \( \projectivedim{Z_1\cap Z_2}\geq 0 \), also \( Z_1\cap Z_2\neq \emptyset \).  
\end{proof}
\file{Projektive Abbildungen}
\section{Projektive Abbildungen}
Sei \( K \) ein Körper, \( V,W \) \( K \)-Vektorraum und \( F\maps V\to W \) eine \( K \)-lineare Abbildung.
\begin{frage*}
  Unter welchen Voraussetzungen induziert \( F \) eine Abbildung \( \projectionspace{V}\to \projectionspace{W} \)?
\end{frage*}
Wir wollen eine Abbildung \( f\maps \projectionspace{V}\to \projectionspace{W}  \) definieren durch
\begin{equation*}
  K\cdot v\mapsto \braceannotate{F(K\cdot v)}{K\cdot F(v)}
\end{equation*}
für \( v\in V\setminus \zeroset \). \( K\cdot F(v) \) ist ein wohldefiniertes Element in \( \projectionspace{W} \) \gdw \( F(v)\neq0 \), \dh wir müssen \( F \) \emph{injektive} voraussetzen.
\begin{definition*}
  Sei \( K \) ein Körper \( V,W \) \( K \)-Vektorräume. Wir nennen ein Abbildung
  \begin{equation*}
    f\maps \projectionspace{V}\to \projectionspace{W}
  \end{equation*}
  \emph{projektiv}, wenn es eine injektive lineare Abbildung \( F\maps \to W \) gibt mit
  \begin{equation*}
    f(K\cdot v)=K\cdot F(v)\quad \forall v\in V\setminus\zeroset.
  \end{equation*}
  Schreibe \( f=\projectionmap{F} \). Ist die projektive Abbildung \( f \) bijektiv, so nennen wir \( f \) \emph{Projektivität}.
\end{definition*}
\begin{bemerkung*}
  Eine projektive Abbildung \( f\maps \projectionspace{V}\to \projectionspace{W}  \) ist immer injektiv.
\end{bemerkung*}
\begin{beispiel*}
  Für \( m\geq n \) betrachte die Einbettung
  \begin{equation*}
    \begin{split}
      F\maps K^{n+1}&\hookrightarrow K^{m+1}\\
      (x_0,\dotsc,x_n)\mapsto (x_0,\dotsc,x_n,0,\dotsc,0).
    \end{split}
  \end{equation*}
  \( F \) induziert eine projektive Abbildung
  \begin{equation*}
    \begin{split}
      f\maps \projectionspaceover{n}{K}&\to \projectionspaceover{n}{K}\\
      (x_0:\dotsc:x_n)&\mapsto(x_0:\dotsc:x_n:0:\dotsc:0).
    \end{split}
  \end{equation*}
  Wir nennen \( f \) die kanonische Einbettung des \( \projectionspaceover{n}{K} \) in den \( \projectionspaceover{m}{K} \).

  \( V=\reals^3 \), \( \ell_0, \ell_1, \ell_2\in \reals[x_0,x_1,x_2] \) linear unabhängige Linearformen in \( x_0,x_1,x_2 \), \dh
  \begin{equation*}
    \ell_i(x_0,x_1,x_2)=\sum_{j=0}^{2}\alpha_{ij}x_j
  \end{equation*}
  mit \( \alpha_{ij\in \reals} \forall i,j\) und \( \det(\alpha_{ij})\neq 0 \). Dann ist \( f\maps \projectionspaceover{2}{\reals} \to \projectionspaceover{2}{\reals}\), \( (x_0:x_1:x_2)\mapsto (\ell_0(\underline{x}):\ell_1(\underline{x}):\ell_2(\underline{x})) \) eine Projektivität der projektiven Ebene über \( \reals \). Als zugehörige lineare Abbildung können wir \zb
  \begin{equation*}
    \begin{split}
      F\maps \reals^3&\to \reals^3\\
    (\underbrace{x_0,x_1,x_2}_{\underline{x}})&\mapsto (l_0(\underline{x}),\ell_1(\underline{x}),\ell_2(\underline{x}))
    \end{split}
  \end{equation*}
  wählen. Die Abbildung
  \begin{equation*}
    F\maps(\underbrace{x_0,x_1,x_2}_{\underline{x}})\mapsto (5l_0(\underline{x}),5\ell_1(\underline{x}),5\ell_2(\underline{x}))
  \end{equation*}
  induziert die gleiche projektive 
  \begin{equation*}
    f=\projectionmap{F}=\projectionmap{F'}.
  \end{equation*}
\end{beispiel*}
\minisec{Allgemein:} Sei \( K \) ein Körper, \( V,W \) \( K \)-Vektorräume, \( F\maps V\to W \) eine injektive lineare Abbildung und \( \lambda\in \fieldwithoutzero{K} \). Dann ist
\begin{equation*}
  \projectionmap{F}=\projectionmap{\lambda F}.
\end{equation*}
\begin{frage*}
  Gibt es \enquote{noch mehr} lineare Abbildungen \( G\maps V\to W \) mit \( \projectionmap{G}=\projectionmap{F} \)?
\end{frage*}
\begin{lemma}
  Notation wie oben. Seien \( F,G\maps V\to W \) lineare injektive Abbildungen mit \( \projectionmap{F}=\projectionmap{G} \). Dann ist \( G=\lambda  \) für ein \( \lambda\in \fieldwithoutzero{K} \).
\end{lemma}
\begin{proof}
  Sei \( \projectionmap{F}=\projectionmap{G} \) und \( v_0\in V\setminus\zeroset \). Dann gilt
  \begin{equation*}
    K\cdot F(v_0)=\projectionmap{F}(Kv_0)=\projectionmap{G}(K\cdot v_0)=K\cdot G(v_0),
  \end{equation*}
  also \texists \( \lambda\in \fieldwithoutzero{K} \) mit \( G(v_0)=\lambda F(v_0) \). Sei \( v\in V\setminus \zeroset \). Wir wollen zeigen, dass gilt
  \begin{equation*}
    G(v)=\lambda F(v).
  \end{equation*}
  \begin{enumerate}[label=Fall \rechtsklammer{\alph*}]
    \item \( v=\alpha v_0 \) mit \( \alpha\in K \). Dann 
    \begin{equation*}
      G(v)=\alpha G(v_0)=\alpha \lambda F(v_0)=\lambda F(v).
    \end{equation*}
    \item \( v \) und \( v_0 \) sind linear unabhängig. Sei
    \begin{equation*}
      G(v)=\mu F(v)\quad \mu\in \fieldwithoutzero{K}
    \end{equation*}
    und
    \begin{equation*}
      G(v+v_0)=\varv F(v+v_0)\quad \varv\in \fieldwithoutzero{K}.
    \end{equation*}
    \( G \) und \( F \) sind linear, also gilt
    \begin{align*}
      0&=G(v+v_0)-G(v)-G(v_0)\\
      &=\varv \braceannotate{F(v)+F(v_0)}{F(v+v_0)}-\mu F(v)-\lambda F(v_0)\\
      0&=(\braceannotate{=0}{\varv-\mu})F(v)+(\braceannotate{=0}{\varv-\lambda})F(v_0).
    \end{align*}
    \( F \) ist injektiv, also sind \( F(v), F(v_0) \)  linear unabhängig. Es folgt 
    \begin{equation*}
      \varv-\mu=\varv-\lambda=0
    \end{equation*}
    und insbesondere \( \mu=\lambda \) \dh
    \begin{equation*}
      G(v)=\lambda F(v)\quad \forall v\in V.
    \end{equation*}
  \end{enumerate}
\end{proof}
\begin{bemerkung*}
  Seien \( V,W \) \( K \)-Vektorräume und \( F \) eine nicht notwendigerweise injektive lineare Abbildung
  \begin{equation*}
    F\maps V\to W.
  \end{equation*}
  Dann ist \( F(K\cdot v) \) für \( v\in V \) genau dann eine Gerade in \( W \) wenn \( F(v)\neq 0 \). Damit induziert \( F \) eine Abbildung
  \begin{equation*}
    \begin{split}
      f\maps \projectionspace{V}\setminus Z\to \projectionmap{W}\\
      K\cdot v\mapsto K\cdot F(v)
    \end{split}
  \end{equation*}
  mit \( Z=\projectionspace{\Ker-{F}} \).
\end{bemerkung*}
\begin{beispiel*}
  Die lineare Abbildung 
  \begin{equation*}
    \begin{split}
      \reals^3&\to \reals^2\\
      (x_0,x_1,x_2)&\mapsto (x_0,x_1)
    \end{split}
  \end{equation*}
  induziert die Abbildung
  \begin{equation*}
    \begin{split}
      p\maps \projectionspaceover{2}{\reals}\setminus\Set{(0:0:1)}&\to \projectionspaceover{1}{\reals}\\
      (x_0:x_1:x_2)\mapsto (x_0:x_1).
    \end{split}
  \end{equation*}
  \begin{erinnerung*}[Beschreibung von affinen Abbildungen in der affinen Geometrie ]
    Seien \( X,Y \) affine Räume über einem Körper \( K \), \( \dim-{X}=n \) und \( p_0,\dotsc,p_n \) affin unabhängige Punkte \( X \). Seien \( q_0,\dotsc,q_n\in Y \). Dann gibt es genau eine affine Abbildung \( f\maps X\to Y \) mit
    \begin{equation*}
      f(p_i)=q_i\quad 0\leq i\leq n.
    \end{equation*}
    Seien \( V,W \) \( K \)-Vektorräume. Auf wie vielen \enquote{unabhängigen} Punkten \( p_i\in \projectionspace{V} \) muss man Bildpunkte \( q_i\in \projectionspace{W} \) vorgeben, \sd eine eindeutig bestimmte projektive Abbildung
    \begin{equation*}
      f\maps \projectionspace{V}\to \projectionspace{W}
    \end{equation*}
    mit \( f(p_i)=q_i \forall i\) besteht.
  \end{erinnerung*}
\end{beispiel*}
\begin{beispiel*}
  \( V=K^{n+1} \). Sei 
  \begin{align*}
    p_0&=(1:0:\dotsc:0)\\
    p_1&=(0:1:\dotsc:0)\\
    &\vdots\\
    p_n&=(0:0:\dotsc:1)
  \end{align*}
  und \( W=V \), \( q_i=p_i\quad \forall 0\leq i\leq n \). Seien \( \lambda_0,\dotsc,\lambda_n\in \fieldwithoutzero{K} \). Dann ist
  \begin{equation*}
    \begin{split}
      f_{(\lambda_0,\dotsc,\lambda_n)}\maps \projectionspaceover{n}{K}&\to \projectionspaceover{n}{K}\\
      (x_0:\dotsc:x_n)&\mapsto (\lambda_0 x_0:\dotsc:\lambda_n x_n)
    \end{split}
  \end{equation*}
  eine Projektivität mit 
  \begin{equation*}
    f_{(\lambda_0,\dotsc,\lambda_n)}(p_i)=q_i
  \end{equation*}
  für \( 0\leq i\leq n \), \emph{aber} unterschiedliche Tupel
   \( (\lambda_0,\dotsc,\lambda_n)\), \( (\mu_0,\dotsc,\mu_n) \) können unterschiedliche Projektivitäten \( f_{(\lambda_0,\dotsc,\lambda_n)} \), \( f_{(\mu_0,\dotsc,\mu_n)} \) induzieren. \Zb ist 
   \begin{equation*}
    (\lambda_0:\dotsc:\lambda_n)=f_{(\lambda_0,\dotsc,\lambda_n)}(1:\dotsc:1)\isittrue{=}f_{(\mu_0,\dotsc,\mu_n)}(1:\dotsc:1)=(\mu_0,\dotsc,\mu_n).
   \end{equation*}
   Das gilt genau dann, wenn \texists \( a\in \fieldwithoutzero{K}  \) mit \( (\lambda_0,\dotsc,\lambda_n)=\alpha (\mu_0,\dotsc,\mu_n) \).
\end{beispiel*}
\begin{idee*}
  Wir legen \( f  \) fest durch die Bilder der \emph{\( n+2 \) Punkte}
  \begin{equation*}
    \equalto{f(p_0)}{q_0},\dotsc,\equalto{f(p_n)}{q_n}
  \end{equation*}
  und \( f((1:\dotsc:1)) \).
\end{idee*}
\file{Projektive Basis}
\begin{definition*}
  Sei \( V \) ein \( K \)-Vektorraum und \( p_0,\dotsc,p_r\in \projectionspace{V} \). Wir nennen das Tupel \( (p_0,\dotsc,p_r) \) \emph{projektiv unabhängig}, wenn es \emph{linear unabhängige} Vektoren \( v_0\dotsc,v_r\in V \) gibt mit \( p_i=K v_i \), \( 0\leq i\leq r \).
\end{definition*}
\begin{bemerkungen*}
  Das Tupel \( (p_0,\dotsc,p_r) \) ist projektiv unabhängig \gdw \( \projectivedim{p_0\vee \dotsb\vee p_r}=r \).
\end{bemerkungen*}
\begin{beispiel*}
  Im \( \projectionspaceover{n}{K} \) sind die Punkte
  \begin{align*}
    p_0&=(1:0:\dotsc:0)\\
    &\vdots\\
    p_n&=(0:0:\dotsc:1)
  \end{align*}
  projektiv unabhängig.
\end{beispiel*}
\begin{definition*}
  Sei \( V \) ein \( K \)-Vektorraum mit \( \dim-{V}=n \) und \( p_0,\dotsc,p_n,p_{n+1}\in \projectionspace{V} \). Wir nennen das \( (n+2) \)-Tupel \( (p_0,\dotsc,p_{n+1}) \) \emph{projektive Basis} von \( \projectionspace{V} \), wenn je \( n+1 \) Punkte davon projektiv unabhängig sind.
\end{definition*}
\begin{beispiel*}
  \( V=K^{n+1} \). Dann sind
  \begin{align*}
    p_0&=(1:0:\dotsc:0)\\
    &\vdots\\
    p_n&=(0:0:\dotsc:1)\\
    p_{n+1}&=(1:\dotsc:1)
  \end{align*}
  eine projektive Basis der \( \projectionspaceover{n}{K} \). Wir nennen \( p_0,\dotsc,p_{n+1}  \) auch kanonische projektive Basis des \( \projectionspace{n}{K} \).
\end{beispiel*}
\begin{lemma}\label{kanonische_projektive_basis_klappt}
  Sei \( V \) ein \( K \)-Vektorraum und \( p_0,\dotsc,p_{n+1} \) eine projektive Basis des \( \projectionspace{V} \). Dann gibt es eine Basis \( v_0,\dots,v_n \) von \( V \), sodass gilt
  \begin{gather*}
    p_i=Kv_i\quad 0\leq i\leq n\\
    p_{n+1}=K(v_0+\dotsb+v_n).
  \end{gather*}
\end{lemma}
\begin{proof}
  \( p_0,\dotsc,p_n \) sind projektiv unabhängig, also gibt es eine Basis \( w_0,\dotsc,w_n \) des \( K \)-Vektorraums \( V \) mit \( p_i=K\cdot w_i \quad 0\leq i\leq n\). Sei \( p_{n+1}=K\cdot w \) mit \( w\in V\setminus\zeroset \). Dann \texists \( \lambda_0,\dotsc,\lambda_n\in K \) mit 
  \begin{equation*}
    w=\lambda_0 w_0+\dotsb+\lambda_n w_n.
  \end{equation*}
\end{proof}
\begin{behauptung*}
  \( \lambda_i\neq 0 \) für \( 0\leq i\leq n \).
\end{behauptung*}
\emph{Denn} angenommen \( \lambda_0=0 \). Dann sind die Vektoren
\begin{equation*}
  w_0,\dotsc,w_{j-1},w_{j+1},\dotsc,w_n,w
\end{equation*}
linear abhängig \contra zu 
\begin{equation*}
  p_0,\dotsc,p_{j-1},p_{j+1},\dotsc,p_n,p_{n+1}
\end{equation*}
projektiv unabhängig. Wähle nun \( v_i=\lambda_i w_i \), \( 0\leq i\leq n \).