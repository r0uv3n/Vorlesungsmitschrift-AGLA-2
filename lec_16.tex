% !TEX root = ./Vorlesungsmitschrift AGLA 2.tex  
\lecture{Di 16.06. 10:15}{}
\begin{proof}[Beweis von \thref{hauptsatz_projektive_geometrie}]
  Im Folgenden sei \( K \) ein Körper, \( V,W \) \( K \)-Vektorräume mit \( \dim-{V}=\dim-{W}\geq 3 \) und \( f\maps \projectionspace{V}\to \projectionspace{W} \) eine Kollineation.
  \begin{lemma}\label{kollineation_erhaelt_verbindungsraum}
    Seien \( p_0,\dotsc,p_r\in \projectionspace{V} \). Dann ist
    \begin{equation*}
      f(p_0 \vee \dotsb\vee p_r)\subseteq f(p_0)\vee \dotsb\vee f(p_r).
    \end{equation*}
  \end{lemma}
  \begin{subproof}
    Induktion über \( r \).
    \begin{proofdescription}
      \item[\( r=0 \)] \( f(p_0)\subseteq f(p_0) \) \checkmark.
      \item[\( r\geq 1 \)] \( p\in (p_0\vee \dotsb\vee p_{r-1})\vee p_r \) mit \( p_i=K\cdot v_i \), \( v_i\in V \), \( 0\leq i\leq r \). 

      Dann ist
      \begin{equation*}
        p_0\vee \dotsb\vee p_{r-1}\vee p_r=\projectionspace{K\cdot v_0+\dotsb+K\cdot v_r}
      \end{equation*}
      und
      \begin{equation*}
        \exists p'\in p_0\vee \dotsb\vee p_{r_1}=\projectionspace{Kv_0+\dotsb+Kv_{r-1}}
      \end{equation*}
      mit \( p\in p'\vee p_r \). Dann gilt
      \begin{align*}
        f(p)&\explain{f \text{ ist Kollineation}}{\in}f(p')\vee (p_r)\\
        &\in f(p_0)\vee \dotsb\vee f(p_{r-1})\vee f(p_r).
      \end{align*}
    \end{proofdescription}
  \end{subproof}
  \begin{lemma}\label{kollineation_verhaelt_sich_gut_mit_basen}
    Sei \( \dim-{V}=\dim-{W}=n+1 \). Dann gibt es Basen \( v_0,\dotsc,v_n \) von \( V \) und \( w_0,\dotsc,w_n \) von \( W \) mit der Eigenschaft
    \begin{equation*}
      f(K\cdot v_i)=K\cdot w_i\quad 0\leq i\leq n 
    \end{equation*}
    und 
    \begin{equation*}
      f(K\cdot (v_0+v_i))=K\cdot (w_0+w_i)1\leq i\leq n.
    \end{equation*}
  \end{lemma}
  \begin{subproof}
    Wähl eine Basis \( v_0,\dotsc,v_n \) von \( V \) und \( w_0',\dotsc,w_n'\in W \), sodass
    \begin{equation*}
      f(K\cdot v_i)=K\cdot w_i'\quad 0\leq i\leq n.
    \end{equation*}
    Es ist
    \begin{align*}
      \projectionspace{W}&=f(\projectionspace{V})\\
      &=f(K\cdot v_0\vee \dotsb\vee K\cdot v_n)\\
      &\explain{\text{\thref{kollineation_erhaelt_verbindungsraum}}}{\subseteq} f(K\cdot v_0)\vee \dotsb\vee f(K\cdot v_n)\\
      &=\projectionspace{K\cdot w_0'+\dotsb+K\cdot w_n'},
    \end{align*}
    also ist \( w_0',\dotsc,w_n' \) eine Basis von \( W \). Es gilt
    \begin{equation*}
      K\cdot (v_0+v_i)\in K\cdot v_0\vee K\cdot v_i,
    \end{equation*}
    also, da \( f \) Kollineation,
    \begin{align*}
      f(K\cdot (v_0+v_i))&\in \braceannotate{K\cdot w_0'}{f(K\cdot v_0)}\vee \braceannotate{K\cdot w_i'}{f(K\cdot v_i)}\\
      &\in K\cdot w_0'\vee K\cdot w_i', 
    \end{align*}
    also \texists \( \lambda_i,\mu_i\in K \), \( 1\leq i \leq n \) mit
    \begin{align*}
      f(K(v_0+v_i))&=K(\mu_i w_0'+\lambda_i w_i')\\
      &=K(w_0'+\inverse{\mu_i}\lambda_i w_i').
    \end{align*}
    Außerdem ist \( \lambda_i,\mu_i\neq 0\quad \forall i \), denn aus \( \lambda_i=0 \), \( \mu_i\neq 0 \) folgt \zb
    \begin{equation*}
      f(K\cdot(v_0+v_i))=K\cdot w_0'=f(K\cdot v_0),
    \end{equation*}
    \contra \( f \) ist bijektiv.

    Also ist \( \lambda_i, \mu_i\in \fieldwithoutzero{K}\quad \forall 1\leq i\leq n \).

    Setze nun \( w_0\definedas w_0' \) und \( w_i=\braceannotate{\in \fieldwithoutzero{K}}{\lambda_i \inverse{\mu_i}}w_i' \), \( 1\leq i\leq n \).
    
  \end{subproof}
  Im Folgenden seien \( v_0,\dotsc,v_n\in V\) und \( w_0,\dotsc,w_n\in W \) wie in \thref{kollineation_verhaelt_sich_gut_mit_basen}, \dh \( v_0,\dotsc,v_n \) ist Basis von \( V \), \( w_0,\dotsc, w_n \) ist Basis von \( W \) mit
  \begin{align*}
    f(K\cdot v_i)&=K\cdot w_i\quad 0\leq i\leq n\\
    f(K\cdot (v_0+ v_i))=K(w_0+w_i)\quad 1\leq i\leq n.
  \end{align*}
  \begin{lemma}\label{kollineation_hat_zugehoerigen_vielleicht_automorphismus}
    Es gibt ein injektive Abbildung
    \begin{equation*}
      \alpha\maps K\to K
    \end{equation*}
    mit \( \alpha(0)=0 \), \( \alpha(1)=1 \), und
    \begin{equation*}
      f(K\cdot (v_0+\lambda v_i))=K\cdot (w_0+\alpha(\lambda)w_i)\quad 1\leq i\leq n\logicspace \forall \lambda\in K.
    \end{equation*}
  \end{lemma}
  \begin{proof}
    Sei \( 1\leq i\leq n \) fest, \( \lambda\in K \). Setze
    \begin{equation*}
      p=K(v_0+\lambda v_i).
    \end{equation*}
    Dann ist \( p\in K\cdot v_0\vee K\cdot v_i \), also \( f(p)\in K\cdot w_0\vee K\cdot w_i \). Aus \( p\neq K\cdot v_i \) folgt \( f(p)\neq K\cdot w_i \) und es gibt \( \alpha_i(\lambda)\in K \) mit 
    \begin{equation*}
      f(p)=K\cdot (w_0+\alpha_i(\lambda)w_i).
    \end{equation*}
    Definiere \( \alpha_i\maps K\to K \), \( \lambda\mapsto \alpha_i(\lambda) \), \( \alpha_i \) ist injektiv, denn für \( \lambda_1\neq \lambda_2 \) ist
    \begin{equation*}
      K\cdot (v_0+\lambda_1 v_i)\neq K\cdot (v_0\lambda_2 v_i).
    \end{equation*}
    Nach Konstruktion von \( v_0,\dotsc,v_n \), \( w_0,\dotsc,w_n \) gilt \( \alpha_i(0)=0 \) und \( \alpha_i(1)=1 \).

    Wir zeigen nun \( \alpha_i=\alpha_j \) für \( 1\leq i,j\leq n \). Seien \( i,j\subset \set{1,\dotsc,n} \), \( i\neq j \). Für \( \lambda\in \fieldwithoutzero{K} \) betrachte
    \begin{equation*}
      p\definedas K\cdot(v_i-v_j)=K\cdot (v_0+\lambda v_i-(v_0+\lambda v_j)).
    \end{equation*}
    Dann ist
    \begin{equation*}
      p\in K\cdot v_i\vee K\cdot v_j
    \end{equation*}
    und
    \begin{equation*}
      p\in K\cdot (v_0 +\lambda v_i)\vee K\cdot (v_0+\lambda v_j),
    \end{equation*}
    also \( f(p)\in K\cdot w_i\vee K\cdot w_j \) und
    \begin{equation*}
      f(p)\in K(w_0+\alpha_i(\lambda)w_i)\vee K\cdot (w_0+\alpha_j(\lambda)w_j).
    \end{equation*}
    Sei \( w\in W \) mit \( f(p)=K\cdot w \). Dann \texists \( \mu_i, \mu_j, \beta_i, \beta_j\in K \) mit
    \begin{align*}
      w&=\mu_i w_i+\mu_j w_j\\
      &=\beta_i (w_0+\alpha_i(\lambda)w_i)+\beta_j (w_0+\alpha_j(\lambda)w_j).
    \end{align*}
    Aus der linearen Unabhängigkeit von \( w_0,w_1,\dotsc,w_n \) folgt
    \begin{equation*}
      \beta_i=-\beta_j\quad \mu_i=\beta_i \alpha_i(\lambda)\quad \mu_j=\beta_j \alpha_j(\lambda),
    \end{equation*}
    also
    \begin{equation*}
      f(p)=K\cdot (\alpha_i(\lambda)w_i-\alpha_j(\lambda)w_j).
    \end{equation*}
    \( p \) ist von \( \lambda\in \fieldwithoutzero{K} \) unabhängig, also
    \begin{align*}
      f(p)&=K\cdot (\alpha_i(1)w_i-\alpha_j(1)w_j)\\
      &=K\cdot(w_i-w_j)\\
      &=K\cdot(\alpha_i(\lambda)w_i-\alpha_j(\lambda)w_j)\quad \forall \lambda\in \fieldwithoutzero{K}.
    \end{align*}
    Also \( \alpha_i(\lambda)=\alpha_j(\lambda) \quad \forall \lambda\in K\).
  \end{proof}
  \begin{lemma}\label{kollineation_vielleicht_automorphismus_verhaelt_sich_gut_mit_komischer_art_basis}
    Notation wie oben. Seien \( \lambda_1,\dotsc,\lambda_n\in K \). Dann ist
    \begin{equation*}
      f(K(v_0+\lambda_1 v_2+\dotsb+\lambda_n v_n))=K\cdot (w_0+\alpha(\lambda_1)w_1+\dotsb+\alpha(\lambda_n)w_n).
    \end{equation*}
  \end{lemma}
  \begin{proof}
    Wir zeigen induktiv für \( 1\leq r\leq n \), dass
    \begin{equation*}
      f(K\cdot (v_0+\lambda_1 v_1+\dotsb+\lambda_r v_r))=K\cdot (w_0+\alpha(\lambda_1)w_1+\dotsb+\alpha(\lambda_r)w_r)\quad \lambda_1,\dotsc,\lambda_r\in K.
    \end{equation*}
    \begin{proofdescription}
      \item[\( r=1 \)] \tto \thref{kollineation_hat_zugehoerigen_vielleicht_automorphismus} \checkmark.
      \item[\( r\geq 2 \)] Sei 
      \begin{equation*}
        p\definedas K\cdot (v_0+\lambda_1 v_1+\dotsb+\lambda_r v_r)
      \end{equation*}
      mit \( \lambda_1,\dotsc,\lambda_r\in K \). Dann ist
      \begin{equation*}
        p\in K(v_0+\lambda_1 v_1+\dotsb+\lambda_{r-1}v_{r-1})\vee K\cdot v_r
      \end{equation*}
      und
      \begin{equation*}
        p\in K(v_0+\lambda_r v_r)\vee K \cdot v_1\vee \dotsb\vee K\cdot v_{r-1},
      \end{equation*}
      also
      \begin{equation*}
        f(p)\in K\cdot (w_0+\alpha(\lambda_1)w_1+\dotsb+\alpha(\lambda_{r-1}w_{r-1}))\vee K\cdot w_r
      \end{equation*}
      und
      \begin{equation*}
        f(p)\in K(w_0+\alpha(\lambda_r)w_r)\vee K\cdot w_1\vee \dotsb\vee K\cdot v_{r-1}.
      \end{equation*}
      Daraus folgt die Existenz von \( \beta,\beta_1,\dotsc,\beta_{r-1}\in K \) mit 
      \begin{align*}
        f(p)&=K\cdot (w_0+\alpha(\lambda_1)w_1+\dotsb+\alpha(\lambda_{r-1})w_{r-1}+\explain{=\alpha(\lambda_r)}\beta\cdot w_r)\\
        &=K\cdot (w_0+\alpha(\lambda_r)w_r+\beta_1 w_1+\dotsb+\beta_{r-1}w_{r-1} \to \beta=\alpha(\lambda_r).
      \end{align*}
    \end{proofdescription}
    
  \end{proof}
  \begin{lemma}\label{kollineation_mit_automorphismus_verhaelt_sich_gut_auf_teilbasis}
    Sei \( (\lambda_1,\dotsc,\lambda_n)\in K^n\setminus \zeroset \). Dann ist 
    \begin{equation*}
      f(K\cdot (\lambda_1 v_1+\dotsb+\lambda_n v_n))=K\nonumber (\alpha(\lambda_1)w_1+\dotsb+\alpha(\lambda_n)w_n).
    \end{equation*}
  \end{lemma}
  \begin{proof}
    Sei \( (\lambda_1,\dotsc,\lambda_n)\neq (0,\dotsc,0) \) und
    \begin{equation*}
      p=K\cdot(\lambda_1 v_1+\dotsb+\lambda_n v_n).
    \end{equation*}
    Es ist
    \begin{equation*}
      f(p)\in K\cdot w_1\vee \dotsb\vee K\cdot w_n
    \end{equation*}
    und
    \begin{equation*}
      f(p)\in K\cdot w_0\vee K\cdot(w_0+\alpha(\lambda_1)w_1+\dotsb+\alpha(\lambda_n)w_n),
    \end{equation*}
    denn
    \begin{equation*}
      K\in Kv_0\vee K(v_0+\lambda_1 v_1+\dotsb+\lambda_n v_n).
    \end{equation*}
    Also \texists \( \beta_1,\dotsc,\beta_n,\beta_0,\beta\in K \) mit
    \begin{equation*}
      f(p)=K\cdot(\beta_1 w_1+\dotsb+\beta_n w_n)
    \end{equation*}
    und
    \begin{equation*}
      f(p)=K\cdot (\explain{-\beta}{\beta_0}w_0+\beta(w_0+\alpha(\lambda_1)w_1+\dotsb+\alpha(\lambda_n)w_n)).
    \end{equation*}
    Es folgt \( \beta_0=-\beta \) und
    \begin{equation*}
      f(p)=K\cdot (\alpha(\lambda_1)w_1+\dotsb+\alpha(\lambda_n)w_n).
    \end{equation*}
      
  \end{proof}
  \begin{lemma}\label{kollineation_vielleicht_automorphismus_ist_automorphismus}
    Die Abbildung \( \alpha\maps K\to K \) aus \thref{kollineation_hat_zugehoerigen_vielleicht_automorphismus} ist ein Körperautomorphismus von \( K \).
  \end{lemma}
  \file{Hauptsatz projektive Geometrie Teil 2}
  \begin{erinnerung*}
    \( \alpha\maps K\to K \) ist injektiv, \( \alpha(0)=0 \), \( \alpha(1)=1 \),
    \begin{equation*}
      f(K\cdot (v_0+\lambda v_i))=K(w_0+\alpha(\lambda)w_i)\quad \forall \lambda\in K\logicspace \forall 1\leq i\leq n.
    \end{equation*}
  \end{erinnerung*}
  \begin{proof}[Beweis von \thref{kollineation_vielleicht_automorphismus_ist_automorphismus}]
    \begin{itemize}
      \item \( \alpha \) ist surjektiv.
      Für \( \mu\in K \) ist 
      \begin{equation*}
        q\definedas K\cdot (w_0+\mu w_1)\in \projectionspace{W},
      \end{equation*}
      also
      \begin{equation*}
        \exists p=K\cdot (\lambda_0 v_0+\dotsb+\lambda_n v_n)\in \projectionspace{V}
      \end{equation*}
      mit \( f(p)=q \), also \( \lambda_0\neq 0 \), daher 
      \begin{equation*}
        p=K\cdot \parens*{v_0+\frac{\lambda_1}{\lambda_0}v_1+\dotsb+\frac{\lambda_n}{\lambda_0}v_n}
      \end{equation*}
      und
      \begin{align*}
        q&=f(p)=K\cdot(w_0+\mu w_1)\\
        &\explain{\text{\thref{kollineation_vielleicht_automorphismus_verhaelt_sich_gut_mit_komischer_art_basis}}}{=}K\cdot \parens*{w_0+\alpha\parens*{\frac{\lambda_1}{\lambda_0}}w_1+\dotsb+\alpha\parens*{\frac{\lambda_n}{\lambda_0}}w_n},
      \end{align*}
      und daher \( \mu=\alpha\parens*{\frac{\lambda_1}{\lambda_0}} \).
      \item Wir zeigen \( \alpha(\lambda+\mu)=\alpha(\lambda)+\alpha(\mu) \quad \lambda,\mu\in K\):
      
      Seien \( \lambda,\mu\in K \). Dann ist
      \begin{equation*}
        p\definedas K\cdot(v_0+(\lambda+\mu)v_1+v_2)\in K\cdot (v_0+\lambda v_1)\vee K(\mu v_1+v_2).
      \end{equation*}
      Also gilt nach Anwendung on \( f \)
      \begin{equation*}
        f(p)\in K\cdot(w_0+\alpha(\lambda)w_1)\vee K(\cdot\alpha(\mu)w_1+w_2),
      \end{equation*}
      also \texists \( \beta,\beta'\in K \) mit
      \begin{equation*}
        w_0+\alpha(\lambda+\mu)w_1+w_2)=\beta(w_0+\alpha(\lambda)w_1)+\beta'(\alpha(\mu)w_1+w_2),
      \end{equation*}
      denn
      \begin{equation*}
        f(p)=K\cdot(w_0+\alpha(\lambda+\mu)w_1+w_2),
      \end{equation*}
      \( w_0,w_1,w_2 \) sind linear unabhängig, also
      \begin{gather*}
        \beta=1=\beta'\\
        \alpha(\lambda+\mu)=\alpha(\lambda)+\alpha(\mu).
      \end{gather*}
      \item Wir zeigen \( \alpha(\lambda\cdot \mu)=\alpha(\lambda)\alpha(\mu) \quad \forall \lambda,\mu\in K\).
      
      Für \( \lambda=0 \) gilt
      \begin{equation*}
        \alpha(0\cdot \mu)=\alpha(0)=0=0\cdot \alpha(\mu)=\alpha(0)\cdot \alpha(\mu).
      \end{equation*}
      Wir können also annehmen, dass \( \lambda\neq 0 \).

      Betrachte
      \begin{equation*}
        p\definedas K\cdot (v_0+\lambda \mu v_1+\lambda v_2)\in K\cdot v_0\vee K(\mu v_1+v_2).
      \end{equation*}
      Also
      \begin{equation*}
        f(p)=K\cdot (w_0+\alpha(\lambda\mu)w_1+\alpha(\lambda)w_2)\in K\cdot w_0\vee K\cdot(\alpha(\mu)w_1+w_2).
      \end{equation*}
      Es gibt also \( \beta,\beta'\in K \) mit
      \begin{equation*}
        w_0+\alpha(\lambda\mu)w_1+\alpha(\lambda)w_2=\beta w_0+\beta'(\alpha(\mu)w_1+w_2).
      \end{equation*}
      Daraus folgt \( \beta=1 \), \( \beta'=\alpha(\lambda) \) und \( \alpha(\lambda\mu)=\braceannotate{\beta'}{\alpha(\lambda)\alpha(\mu)} \).
    \end{itemize}
  \end{proof}
  \begin{lemma}
    Sei
    \begin{equation*}
      (\lambda_0,\dotsc,\lambda_n)\in K^{n+1}\setminus \Set{(0,\dotsc,0)}.
    \end{equation*}
    Dann ist
    \begin{equation*}
      f(K\cdot (\braceannotate{\in V}{\lambda_0 v_0+\dotsb+\lambda_n v_n}))=K\cdot(\alpha(\lambda_0)w_0+\dotsb+\alpha(\lambda_n)w_n).
    \end{equation*}
  \end{lemma}
  \begin{proof}
    Ist \( \lambda_0=0 \), so verwende \thref{kollineation_mit_automorphismus_verhaelt_sich_gut_auf_teilbasis}. Wir können also \( \lambda\neq 0 \) annehmen. Nach \thref{kollineation_vielleicht_automorphismus_verhaelt_sich_gut_mit_komischer_art_basis}
    \begin{align*}
      f(K\cdot (\lambda_0 v_0+\lambda_1 v_1+\dotsb+\lambda_n v_n))&= f\parens*{K\cdot \parens*{v_0+\frac{\lambda_1}{\lambda_0}v_1+\dotsb+\frac{\lambda_n}{\lambda_0}v_n}}\\
      &= K\cdot \parens*{w_0+\alpha\parens*{\frac{\lambda_1}{\lambda_0}}w_1+\dotsb+\alpha\parens*{\frac{\lambda_n}{\lambda_0}}w_n}\\
      &\explain{\text{\thref{kollineation_vielleicht_automorphismus_ist_automorphismus}}}{=}K\cdot \parens*{w_0+\frac{\alpha(\lambda_1)}{\alpha(\lambda_0)}w_1+\dotsb+\frac{\alpha(\lambda_n)}{\alpha(\lambda_0)}w_n}\\
      &=K\cdot(\alpha(\lambda_0 )w_0+\alpha(\lambda_1)w_1+\dotsb+\alpha(\lambda_n)w_n).
    \end{align*}
  \end{proof}
  Die Abbildung
  \begin{equation*}
    \begin{split}
      F\maps V\to W\\
      \lambda_0 v_0+\dotsb+\lambda_n v_n&\mapsto \alpha(\lambda_0)w_0+\dotsb+\alpha(\lambda_n)w_n
    \end{split}
  \end{equation*}
  ist semilinear und injektiv und es gilt \( f=\projectionmap{F} \). Damit ist \( f \) eine Semiprojektivität und \thref{hauptsatz_projektive_geometrie} bewiesen.
\end{proof}